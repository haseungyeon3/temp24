\documentclass[letterpaper,10pt]{report}
\usepackage{colortbl}
\usepackage{multirow}
\usepackage{amsmath,amssymb,amsfonts,epsfig,graphicx}
\usepackage{textcomp}
\usepackage{fancyhdr}
\usepackage{setspace}
\usepackage[toc,page]{appendix}
\usepackage{multirow}

\usepackage{tabularx} 
\usepackage{multicol}
\usepackage{rotating}
\usepackage{subfig}
\usepackage{url}
\usepackage{soul}
\usepackage{color}
\usepackage{float}
\usepackage{enumerate}
\usepackage{xcolor}
\usepackage{fancybox}
\usepackage{tikz}
\usepackage[margin=0.75in]{geometry}
\usepackage{multicol}
\usepackage{hyperref}
\usepackage{blindtext}

\usepackage{lipsum,fancyhdr,lastpage,refcount}
 
 
\definecolor{myBlue}{rgb}{0,0.35,0.7}
\hypersetup{
    colorlinks=true,
    linkcolor=red,
    citecolor=green,
    filecolor=magenta,
    urlcolor=myBlue
}

\newcommand{\webURL}[1]{\urlstyle{same}{ \textit{\textcolor{darkgray}{\url{#1}}}}}
\newcommand{\webLink}[2]{\href{#1}{\textcolor{ darkgray}{{#2}}}}
\newcommand{\appLink}[2]{\href{#1}{\textcolor{white}{{#2}}}}

\pdfpagewidth 8.5in
\pdfpageheight 11in

\pagestyle{fancy} 
% Define box and box title style
\tikzstyle{mybox} = [draw=gray, fill=gray!20, very thick,
    rectangle, rounded corners, inner sep=10pt, inner ysep=20pt]
\tikzstyle{fancytitle} =[fill=black!80, text=white]

\begin{document}
\pagenumbering{arabic}
\pagestyle{fancy}
%\fancyhf{} % clear existing header/footer entries
% Place Page X of Y on the right-hand
% side of the footer
%\fancyfoot[R]{Page \thepage}


% _____ copyright and usage information _____ %
 

% _____ set page headers and footers _____ %

\lhead{TEXAS A\&M. STAT }
\begin{center}
\ \\[3mm]
\large{STATISTICS 624; Fall 2024 } \\[2mm]
\large{Section  ;  } \\[5mm]
\end{center}
 \vspace{5mm}
 \footnotesize
\textbf{This syllabus is tentative and subject to change:} Any updates or adjustments will be communicated during class sessions and/or posted on the course website. The course website is accessible at https://canvas.tamu.edu. It is the student's responsibility to consistently check the course website for updates.\\
\:\
\\
\normalsize
\textbf{Course Information} 
\begin{itemize}
  \setlength\itemsep{0.05em}

\item Course Number:  
\item Time :      
\item Location:  
\item Credit Hours: 3.0
 \end{itemize}
\:\
\\
\textbf{
Instructor Details}
\begin{itemize}  \setlength\itemsep{0.05em}

\item  Instructor:  Seung-Yeon "Shawn" Ha, Ph.D

\item Office/Phone:        Blocker 245E, (979) 845-3141
\item E-Mail:                      \webURL{ha.shawn@stat.tamu.edu}

(The email subject must include \textcolor{blue}{your course name}. Please do not use the email address linked in Canvas mailing system.)


\item Office Hours:  
\item Other times and in-person meetings can be arranged upon request. 
\end{itemize} 
\:\
\\
\textbf{Teaching Assistant Information}  
 \begin{itemize}\setlength\itemsep{0.05em}
 \item   
 \end{itemize}
 
\:\
\\
\textbf{Important dates}:\webURL{https://registrar.tamu.edu/Academic-Calendar} 

\begin{itemize}
   \item Exam Dates and Times
   \begin{itemize}
      \setlength\itemsep{0.05em}
   \item Midterm Exam:  
   \item Final Exam:  
   \end{itemize}
   \item University Dates and Times
   \begin{itemize}    \setlength\itemsep{0.05em}
   \item Last day for adding/dropping courses: Friday, August 23rd, 2024
   \item Faculty and staff holiday:  Monday, September 2nd, 2024
   \item No class due to an university event (SCC): Friday, September 27th, 2024
   \item Fall break: Monday, October 7th, 2024
   \item Last day to drop courses with no penalty (Q-drop): 5pm, Wednesday, November 13th, 2024
 
      \item Thanksgiving Holiday:  November 27th-November 29th, 2024
   \item Last day of class: Monday, December 2nd, 2024
   \end{itemize}
\end{itemize}
\clearpage

\noindent \textbf{Course	Description}\\ 
\textbf{Course	Prerequisites } \\ 
\\
\:\
\\ 
\textbf{Course Learning Outcomes}  
\begin{itemize} \setlength\itemsep{0.05em}
\item  
\end{itemize}
\:\
\\
\textbf{Textbook and Resource Materials}  
\textbf{Grading	 } 
\begin{itemize}
\item Course percentage performance (PP) yields grades as follows. 
\item[] A: 90\% $\le$ PP $\le$ 100\%; B: 
80\% $\le$ PP $<$ 90\%; C: 70\% $\le$ PP $<$ 80\%; D: 60\% $\le$ PP $<$70\%; F: 0\% $\le$ PP $<$ 60\%.
\item  
\item \textbf{Incomplete:}\\ 
\end{itemize} 
\clearpage


\begin{center}\textbf{University Policies} \end{center}
\textbf{Copyright Notice}\\
Faculty members own copyright in their educational work at Texas A \& M University, as stated in 
the \href{https://policies.tamus.edu/17-01.pdf}{Texas A\& M University System Policy for Intellectual Property Management and 
Commercialization}. Students are not allowed to post or share any materials created by a faculty 
member unless given permission by that faculty member. This includes but is not limited to 
homework assignments, homework solutions, exams, exam solutions, lecture notes and any other 
supplemental materials. \textbf{Any violation of this copyright policy could result in disciplinary 
actions as described in Student Rule 20.2: Procedures in Scholastic Dishonesty Cases and 
Student Rule 20.1.2.3.1} 
\\
\:\
\\
\textbf{Americans	with	Disabilities	Act	(ADA)	Policy}  \\
 If	 you	 experience	 barriers	 to	 your	 education	 due	 to	 a	 disability	 or	 think	 you	 may	 have	 a	
disability,	please	contact	Disability	Resources	office	on	your	campus	(resources	listed	below).	Disabilities	
may	 include,	 but	 are	 not	 limited	 to	 attentional,	 learning,	 mental	 health,	 sensory,	 physical,	 or	 chronic	
health	 conditions.	 All	 students	 are	 encouraged	 to	 discuss	 their	 disability	 related	 needs	 with	 Disability	
Resources	and	their	instructors	as	soon	as	possible.		Disability	Resources	is	located	in	the	Student	Services	
Building	or	at	(979)	845-1637	or	visit	disability.tamu.edu. Once you are approved by them, you can schedule the exams at AIM that will provide you the accommodations that you need during the tests. \webURL{https://cascade.accessiblelearning.com/TAMU}\\
\:\
\\
\textbf{Attendance	Policy}\\
The	university	views	class	attendance	and	participation	as	an	individual	 student	 responsibility.	Students	
are	expected	to	attend	class	and	to	complete	all	assignments.
Please	 refer	 to	 Student	 Rule	 7 in	 its	 entirety	 for	 information	 about	 excused	 absences,	 including	
definitions,	and	related	documentation	and	timelines.
Makeup	Work	Policy
Students	 will	 be	 excused	 from	 attending	 class	 on	 the	 day	 of	 a	 graded	 activity	 or	 when	 attendance	
contributes	 to	 a	 student’s	 grade,	 for	 the	 reasons	 stated	 in	 Student	 Rule	 7,	 or	 other	 reason	 deemed	
appropriate	by	the	instructor.
Please	 refer	 to	 Student	 Rule	 7 in	its	entirety	 for	information	about	makeup	work,	including	 definitions,	
and	related	documentation	and	timelines.
Absences	related	to	Title	IX	of	the	Education	Amendments	of	1972	may	necessitate	a	period	of	more	than	
30	days	for	make-up	work,	and	the	timeframe	 for	make-up	work	should	be	agreed	upon	by	the	student	
and	instructor”	(Student	Rule	7,	Section	7.4.1).
“The	instructor	is	under	no	obligation	to	provide	an	opportunity	for	the	student	to	make	up	work	missed	
because	of	an	unexcused	absence”	(Student	Rule	7,	Section	7.4.2).
Students	 who	 request	 an	 excused	 absence are	 expected	 to	 uphold	 the	 Aggie	 Honor	 Code	 and	 Student	
Conduct	Code.	(See	Student	Rule	24).\\
Here	are	the	examples	of	excused	absence:
\begin{itemize}
  \setlength\itemsep{0.05em}
\item Death	or	major	illness	in	a	student’s	immediate	family.
\item  Illness	of	a	dependent	family	member.
\item  Participation	in	legal	proceedings	or	Religious	holy	day
\item  Illness	that	is	too	severe	or	contagious	for	the	student	to	attend	class
\item  Required	participation	in	military	duties
\item  Mandatory	admission	interviews	for	professional	or	graduate	school

\end{itemize}
If	 the	student	is	seeking	an	excused	absence,	 the	student	must	notify	 the	instructor	as	soon	as	possible	
after	the	absence,	but	no	later	that	the	end	of	the	second	working	day	after	the	last	day	of	the	absence.
The	student	is	responsible	for	providing	satisfactory	evidence	to	the	instructor	to	substantiate	the	reason	
for	absence.	If	the	absence	was	excused,	the	instructor	must	either	provide	the	student	an	opportunity	to	
make	up	the	exam	or	other	missed	work, or	provide	a	satisfactory	alternative.\\
\:\
\\
\textbf{Academic	Integrity	Statement	and	Policy}\\
``An	Aggie	does	not	lie,	cheat,	steal,	or	tolerate	those	who	do."\\
“Texas A\& M University students are responsible for authenticating all work submitted to an 
instructor. If asked, students must be able to produce proof that the item submitted is indeed the 
work of that student. Students must keep appropriate records at all times. The inability to 
authenticate one’s work, should the instructor request it, may be sufficient grounds to initiate an 
academic misconduct case” (Section 20.1.2.3, Student Rule 20).
You can learn more about the Aggie Honor System Office Rules and Procedures, academic 
integrity, and your rights and responsibilities at  
\webURL{http://aggiehonor.tamu.edu}.	
\\
\:\
\\
\textbf{Scholastic Dishonesty}\\
It is the responsibility of both Students and Instructors to help maintain scholastic integrity at the 
university by refusing to participate in or tolerate scholastic dishonesty. Any violation of 
scholastic dishonesty could result in disciplinary actions as described in Student Rule 20. \\
Cheating: Intentionally using or attempting to use unauthorized materials, information, notes, 
study aids or other devices or materials in any academic exercise. Unauthorized materials may 
include anything or anyone that gives a student assistance and has not been specifically approved 
in advance by the instructor. Some examples of scholastic dishonesty are given here:
\begin{itemize}   \setlength\itemsep{0.05em}

\item  Improper Acquiring of Information: Acquiring answers for any assigned work or 
examination from any unauthorized source. Working with another person or persons on 
any assignment or examinations when not specifically permitted by the instructor.
Observing the work of other students during any examination both in class exams and 
take home exams.
\item  Providing Information: Providing answers for any work or examination when not 
specifically authorized to do so. Informing any person or persons of the contents of any 
examination prior to the time the examination is given.
\item  Plagiarism: Failing to credit sources used in a work product (homework, take home 
examination, paper, dissertation) in an attempt to take credit for the work of someone 
else. Attempting to receive credit for work performed by another person, including papers 
obtained in whole or in part from individuals or other sources.
\item  Conspiracy: Agreeing with one or more persons to commit any act of scholastic 
dishonesty.
\item  Fabrication of Information: The falsification of the results obtained from a research or 
laboratory experiment. The written or oral presentation of results of research or 
laboratory experiments without the research or laboratory experiment having been 
performed.
\item  Violation of Computer Use: Violation of any announced departmental or college rule 
relating to academic matters, including but not limited to abuse or misuse of computer 
access or information.
\item  Aggie Honor System: If you encounter students cheating or not abiding by university 
rules, then it is mandatory that you report the student to the Aggie Honor System Office: 
complete information is at \webURL{aggiehonor.tamu.edu}.
 \end{itemize}
\:\
 \\
\textbf{	Title	IX	and	Statement	on	Limits	to	Confidentiality:}\\
Texas	A\& M	University	is	committed	to	fostering	a	learning	environment	that	is	safe	and	productive	for	all.	
University	 policies	 and	 federal	 and	 state	 laws	 prohibit	 gender-based	 discrimination	 and	 sexual	
harassment,	including	sexual	assault,	sexual	exploitation,	domestic	violence,	dating	violence,	and	stalking.
With	the	exception	of	some	medical	and	mental	health	providers,	all	university	employees	(including	full	
and	part-time	faculty,	staff,	paid	graduate	assistants,	student	workers,	etc.)	are	Mandatory	Reporters	and	
must	report	to	the	Title	IX	Office	if	the	employee	experiences,	observes,	or	becomes	aware	of	an	incident	
that	meets	the	following	conditions (see	University	Rule	08.01.01.M1):
The	incident	is	reasonably	believed	to	be	discrimination	or	harassment.
The	incident	is	alleged	to	have	been	committed	by	or	against	a	person	who,	at	the	time	of	the	incident,	
was	(1)	a	student	enrolled	at	the	University	or	(2)	an	employee	of	the	University.
Mandatory	 Reporters	 must	 file	 a	 report	 regardless	 of	 how	 the	 information	 comes	 to	 their	 attention	 –
including	 but	 not	 limited	 to	 face-to-face	 conversations,	 a	 written	 class	 assignment	 or	 paper,	 class	
discussion,	email,	 text,	 or	 social	media	 post.	Although	Mandatory	 Reporters	must	 file	a	 report,	in	most	
instances,	 a	 person	 who	 is	 subjected	 to	 the	 alleged	 conduct	 will	 be	 able	 to	 control	 how	 the	 report	 is	
handled,	including	whether	or	not	to	pursue	a	formal	investigation.	The	University’s	goal	is	to	make	sure	
you	are	aware	of	the	range	of	options	available	to	you	and	to	ensure	access	to	the	resources	you	need.
Texas	A\&M	at	College	Station
Students	wishing	 to	 discuss	 concerns	in	a	 confidential	 setting	are	encouraged	 to	make	an	appointment	
with	Counseling	and	Psychological	Services (CAPS).
Students	can	learn	more	about	filing	a	report,	accessing	supportive	resources,	and	navigating	the	Title	IX	
investigation	and	resolution	process	on	the	University’s	Title	IX	webpage.	\\
\:\
\\
\textbf{Statement	on	Mental	Health	and	Wellness}\\
Texas	 A\&M	 University	 recognizes	 that	 mental	 health	 and	 wellness	 are	 critical	 factors	 that	 influence	 a	
student’s	academic	success	and	overall	wellbeing.	Students	are	encouraged	to	engage	in	healthy	self-care	
by	utilizing	available	resources	and	services	on	your	campus.
Texas	A\&M	College	Station
Students	who	need	someone	to	talk	to	can	contact	Counseling	\&	Psychological	Services (CAPS)	or	call	the	
TAMU	Helpline (979-845-2700)	from	4:00	p.m.	to	8:00	a.m.	weekdays	and	24	hours	on	weekends.	24-hour	
emergency help	 is	 also	 available	 through	 the	 National	 Suicide	 Prevention	 Hotline	 (800-273-8255)	 or	 at	
suicidepreventionlifeline.org.	\\
\:\
\\
\textbf{Statement on the Family Educational Rights and Privacy Act (FERPA)}\\
FERPA is a federal law designed to protect the privacy of educational records by limiting access 
to these records, to establish the right of students to inspect and review their educational records 
and to provide guidelines for the correction of inaccurate and misleading data through informal 
and formal hearings. Currently enrolled students wishing to withhold any or all directory 
information items may do so by going to howdy.tamu.edu and clicking on the "Directory Hold 
Information" link in the Student Records channel on the MyRecord tab. The complete FERPA 
Notice to Students and the student records policy is available on the Office of the Registrar 
webpage.\\
Items that can never be identified as public information are a student’s social security number, 
citizenship, gender, grades, GPR or class schedule. All efforts will be made in this class to 
STAT 624 Syllabus Fall 2023
Page 13 of 14
protect your privacy and to ensure confidential treatment of information associated with or 
generated by your participation in the class.
Directory items include name, UIN, local address, permanent address, email address, local 
telephone number, permanent telephone number, dates of attendance, program of study (college, 
major, campus), classification, previous institutions attended, degrees honors and awards 
received, participation in officially recognized activities and sports, medical residence location 
and medical residence specialization.
\\
\:\
\\
\textbf{Tell Somebody}\\
Oftentimes after a tragedy, people come forward with information and observations that, in 
retrospect, may have signaled a larger issue. This information when viewed collectively may be 
helpful in preventing tragic events and initiating assistance to an individual. Texas A\& M 
University is committed to a proactive approach and needs your help. \\
As a member of this University community, if you observe any behavior that is concerning you 
may report the behavior using the online report form or by contacting one of the Special 
Situations Team members during business hours. The Special Situations Team is comprised of 
University faculty and staff charged with helping students, faculty and staff who are exhibiting 
concerning behavior.\\
This is not a system to be used for emergencies. If you are in an emergency situation that 
requires medical, psychological or police services, call 911.
\clearpage

  \textbf{Tentative Course Schedule}\\ 
  
\end{document}



